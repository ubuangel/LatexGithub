\documentclass[12pt]{article}
\usepackage{caption}

\usepackage[utf8]{inputenc}
\usepackage[T1]{fontenc}

\usepackage{enumerate}
\usepackage[spanish]{babel}
\usepackage{titlesec}
\usepackage{titling}
\usepackage{xcolor}
\usepackage{hyperref}%compilar 2 veces
\usepackage{geometry}
\usepackage{listings}
\usepackage{pdfpages}
\usepackage{subfig}%figuras una alado de otraa
\usepackage{float}
\definecolor{commentsColor}{rgb}{0.13, 0.55, 0.13}
\definecolor{keywordsColor}{rgb}{0.000000, 0.000000, 0.635294}
\definecolor{stringColor}{rgb}{0.558215, 0.000000, 0.135316}
\definecolor{numerolineas}{rgb}{0.41,0.41,0.41}

\usepackage{verbatim}%coemntario begin{} end

\usepackage{fancyhdr}%activar para usar encabezados esttilos funcy


\usepackage[backend=bibtex]{biblatex}
\addbibresource{referencias/referencias.bib}

%\usepackage[top=1.5cm,bottom=1.0cm,left=1.25cm,right=1.25cm]{geometry}%para todos las paginas


\lstset{
        tabsize=2, % tab = 2 espacios
        backgroundcolor=\color[HTML]{F0F0F0}, % color de fondo
        captionpos=b, % posición de pie de código, b=debajo
        basicstyle=\ttfamily, % estilo de letra general
        columns=fixed, % columnas alineadas
        extendedchars=true, % ASCII extendido
        breaklines=true, % partir líneas
        prebreak = \raisebox{0ex}[0ex][0ex]{\ensuremath{\hookleftarrow}}, % marcar final de línea con flecha
        showtabs=false, % no marcar tabulación
        showspaces=false, % no marcar espacios
        keywordstyle=\bfseries\color[HTML]{007020}, % estilo de palabras clave
        commentstyle=\itshape\color[HTML]{60A0B0}, % estilo de comentarios
        stringstyle=\color[HTML]{4070A0}, % estilo de strings
}

\lstdefinestyle{abaqusPython}{
        language=python,
		% Palabras clave extra
        morekeywords={CONTINUOUS,NUMBER,MESH,par,name,ParStudy,
		template,define,sample,combine, generate},
		% Delimitadores extra, s porque hay uno a cada lado
        moredelim=[s][\ttfamily\color{magenta}]{<}{>},
}



\begin{document}
%\newgeometry{bottom=2.5cm,top=2.6cm,left=2.5cm,right=2.5cm}
%\restoregeometry
%\includepdf{9}

%
%para el report 

\begin{titlepage}

	\centering
	\includegraphics[width=0.15\textwidth]{imagenes/unsa.png}\par\vspace{1cm}
	{\scshape\LARGE Universidad Nacional de San Agustin \par}
	\vspace{1cm}
	{\scshape\Large Proyecto \par}
	\vspace{1.5cm}
	{\huge\bfseries Titulo\par}
	\vspace{2cm}
	{\Large\itshape Bejar Merma Angel Andres\par}
	\vfill
	Docente \par
	Dr. Nombre docente \textsc{}

	\vfill

% Bottom of the page
	{\large \today\par}
\end{titlepage}

%para el report





%\begin{abstract}



%\end{abstract}

%\vspace{5mm} %espacio vertical usar en imagen



\section{Parte 1}

\begin{enumerate}[(a)]

\item 
\item 

\item 
\item 

\item 
\end{enumerate}


\begin{figure}[H]%estricto
\centering
%\includegraphics[width=15cm]{imagenes/.png}
\caption{}
\label{fig:1}
\end{figure}

%\lstinputlisting[style=abaqusPython]{codigo.py}


\begin{figure}[H]%estricto
\centering
%\includegraphics[width=15cm]{imagenes/.png}
\caption*{}

\end{figure}
\newpage
\section{Desarrollo}
\newpage



\section{Codigo}


%\end{samepage}









\newpage
\section{Conclusion}




	


\vspace{20 mm}






\printbibliography

\end{document}

